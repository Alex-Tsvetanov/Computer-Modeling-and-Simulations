
\section{Постановка на задачата}
Целта на упражнението е да се изследват характеристиките на компютърни системи чрез симулационно моделиране в средата GPSS World. Задачите включват:
\begin{enumerate}
    \item Симулиране на еднопроцесорна система и изследване на натоварването при вариране на интензивността на входния поток.
    \item Изследване на статистическите свойства на генератор на случайни събития.
    \item Симулиране на системи с приоритетно обслужване и с ограничена буферна памет.
\end{enumerate}

\section{Резултати от експериментите}

\subsection{Задача 3: Еднопроцесорна система (M/G/1)}
Изследвана е еднопроцесорна система с равномерно разпределено време за обслужване $5 \pm 3$ s. Входният поток е с експоненциално разпределение, като интензивността $\lambda$ варира.

\textbf{Код на модела (GPSS):}
\begin{lstlisting}[language=bash]
; Task 3: Single Server M/G/1
EXPON FUNCTION RN1,C24
0,0/.1,.104/.2,.222/.3,.355/.4,.509/.5,.69/.6,.915/.7,1.2
.75,1.38/.8,1.6/.84,1.83/.88,2.12/.9,2.3/.92,2.52/.94,2.81
.95,2.99/.96,3.2/.97,3.5/.98,3.9/.99,4.6/.995,5.3/.998,6.2
.999,7/.9998,8

    GENERATE 40,FN$EXPON   ; Mean varies: 40, 20, 13.3, 10, 6.6, 5
    QUEUE    Line
    SEIZE    Server
    DEPART   Line
    ADVANCE  5,3
    RELEASE  Server
    TERMINATE 1
    START    1000
\end{lstlisting}

\subsubsection{Резултати}

\begin{table}[H]
    \centering
    \caption{Резултати от експериментите за Задача 3}
    \begin{tabular}{|c|c|c|}
        \hline
        \textbf{Интензивност} $\lambda$ [заявки/s] & \textbf{Среден интервал} [s] & \textbf{Натовареност} (Utilization) \\
        \hline
        0.025 & 40 & 0.124 \\
        \hline
        0.050 & 20 & 0.241 \\
        \hline
        0.075 & 13.33 & 0.369 \\
        \hline
        0.100 & 10 & 0.490 \\
        \hline
        0.150 & 6.67 & 0.766 \\
        \hline
        0.200 & 5.00 & 0.944 \\
        \hline
    \end{tabular}
\end{table}

\begin{figure}[H]
    \centering
    \begin{tikzpicture}
        \begin{axis}[
            title={Натовареност на процесора спрямо Интензивност},
            xlabel={Интензивност $\lambda$ [заявки/s]},
            ylabel={Натовареност (Utilization)},
            grid=major,
            width=0.8\linewidth,
            height=6cm,
            xmin=0, xmax=0.22,
            ymin=0, ymax=1.0,
            xtick={0, 0.025, 0.05, 0.075, 0.1, 0.15, 0.2},
            xticklabels={0, 0.025, 0.05, 0.075, 0.1, 0.15, 0.2},
            x tick label style={rotate=45, anchor=east},
        ]
        \addplot[
            color=blue,
            mark=*,
            thick
            ]
            coordinates {
            (0.025, 0.124)(0.050, 0.241)(0.075, 0.369)(0.100, 0.490)(0.150, 0.766)(0.200, 0.944)
            };
        \end{axis}
    \end{tikzpicture}
    \caption{Графика на зависимостта Натовареност = f(Интензивност)}
    \label{fig:task3_plot}
\end{figure}

\subsection{Задача 4: Изследване на генератор на заявки}
Генерирани са 1000 заявки с експоненциално разпределение за проверка на статистическите характеристики.

\textbf{Код на модела (GPSS):}
\begin{lstlisting}[language=bash]
IAT_TAB TABLE P1,0,2,20  ; Table using P1 (Calculated Inter-Arrival)

    GENERATE 10,FN$EXPON
    ASSIGN    1,AC1          ; P1 = Current Time
    ASSIGN    1-,X$LAST_TIME ; P1 = Current - Last (Interval)
    SAVEVALUE LAST_TIME,AC1  ; Update Last Time
    TABULATE  IAT_TAB        ; Record Interval
    TERMINATE 1
    START    1000
\end{lstlisting}

\begin{figure}[H]
    \centering
    \includegraphics[width=0.8\linewidth]{images/task4.png}
    \caption{Хистограма на разпределението на интервалите}
    \label{fig:task4_hist}
\end{figure}

\subsection{Задача 5а: Система с два потока и приоритети}
Симулирана е система с два входни потока:
\begin{itemize}
    \item \textbf{Поток 1 (Висок приоритет):} Интервал $60 \pm 20$, Обслужване $50 \pm 20$.
    \item \textbf{Поток 2 (Нисък приоритет):} Интервал $80 \pm 30$, Обслужване $100 \pm 30$.
\end{itemize}

\textbf{Код на модела (GPSS):}
\begin{lstlisting}[language=bash]
    GENERATE 60,20,,,2    ; High Priority
    QUEUE    WaitLine
    SEIZE    Processor
    DEPART   WaitLine
    ADVANCE  50,20
    RELEASE  Processor
    TERMINATE 0

    GENERATE 80,30,,,1    ; Low Priority
    QUEUE    WaitLine
    SEIZE    Processor
    DEPART   WaitLine
    ADVANCE  100,30
    RELEASE  Processor
    TERMINATE 0
\end{lstlisting}

\begin{figure}[H]
    \centering
    \includegraphics[width=0.8\linewidth]{images/task5a.png}
    \caption{Резултати от симулацията на Задача 5а}
    \label{fig:task5a_results}
\end{figure}

\subsection{Задача 5б: Система с ограничен буфер}
Системата има буфер с капацитет 10 заявки. При запълване на буфера, новите заявки се отхвърлят.
\begin{itemize}
    \item Входен поток: Експоненциален (Mean=80).
    \item Обслужване: $60 \pm 40$ s.
\end{itemize}

\textbf{Код на модела (GPSS):}
\begin{lstlisting}[language=bash]
BUF STORAGE 10
    GENERATE 80,FN$EXPON
    TRANSFER BOTH,TRY_BUF,DROP 
TRY_BUF ENTER    BUF
    SEIZE    Processor
    LEAVE    BUF
    ADVANCE  60,40
    RELEASE  Processor
    TERMINATE 0
DROP TERMINATE 0
\end{lstlisting}

\begin{figure}[H]
    \centering
    \includegraphics[width=0.8\linewidth]{images/task5b.png}
    \caption{Резултати от симулацията на Задача 5б}
    \label{fig:task5b_results}
\end{figure}

\section{Анализ и Заключение}
Проведените експерименти демонстрират основните принципи на работа със системата GPSS.
\begin{itemize}
    \item В \textbf{Задача 3} се наблюдава линейна зависимост между натовареността на процесора и интензивността на входния поток, докато системата не наближи насищане.
    \item В \textbf{Задача 4} хистограмата потвърждава експоненциалния характер на генерирания поток.
    \item В \textbf{Задача 5} се вижда влиянието на приоритезацията върху времето за чакане на различните класове заявки, както и ефектът на ограничения буфер върху загубата на заявки.
\end{itemize}
