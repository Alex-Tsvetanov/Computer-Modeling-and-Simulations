\section{Постановка на задачата}
Съобщенията постъпват в изчислителната система от два типа сензори – тип А и тип B.
\begin{itemize}
    \item Съобщенията от тип А пристигат през интервали със средна стойност $(9 \pm 4)$ секунди.
    \item Съобщенията от тип B се получават на всеки 2 секунди.
    \item Преди да бъдат обработени, съобщенията се групират в задачи, като всяка задача се състои от 4 съобщения.
    \item Всяка задача преминава през предварителна обработка, която се извършва от два процесора, като задачите се разпределят на случаен принцип между тях.
    \item Процесът на предварителна обработка отнема 10 секунди.
    \item В 24\% от случаите обработката е неуспешна и задачите трябва да бъдат обработени повторно.
    \item Успешно обработените задачи се предават към разпределителен модул, където се разделят.
    \item Разпределителят събира по 2 съобщения от тип А и 3 съобщения от тип B, след което ги записва едно по едно в хранилището.
    \item Всяко съобщение се съхранява едновременно както в основно, така и в резервно хранилище.
\end{itemize}
Необходимо е да се симулира работата на тази система за период от 4 часа.

\section{Теоретична част}
Системата може да се разглежда като система за масово обслужване с групиране на съобщения и повторни опити при неуспех.

Входящите потоци:
\begin{itemize}
    \item \textbf{Сензор A:} Равномерно разпределение в интервала $[5, 13]$ секунди (средно 9 секунди).
    \item \textbf{Сензор B:} Фиксиран интервал от 2 секунди.
\end{itemize}

Очакван брой съобщения за 4 часа:
\begin{itemize}
    \item Сензор A: $\frac{4 \times 3600}{9} \approx 1600$ съобщения.
    \item Сензор B: $\frac{4 \times 3600}{2} = 7200$ съобщения.
    \item Общо: около 8800 съобщения.
\end{itemize}

Брой задачи: $\frac{8800}{4} = 2200$ задачи.

Обработка:
\begin{itemize}
    \item Предварителна обработка: 10 секунди на задача.
    \item Неуспешни обработки: $2200 \times 0.24 = 528$ задачи.
    \item Повторни обработки: 528 задачи.
    \item Общо време за обработка: $(2200 + 528) \times 10 = 27280$ секунди.
\end{itemize}

Капацитет на разпределителния модул:
\begin{itemize}
    \item За една дистрибуция са необходими 2 съобщения от тип A и 3 от тип B.
    \item Максимален брой дистрибуции: $\min(\frac{1600}{2}, \frac{7200}{3}) = \min(800, 2400) = 800$.
\end{itemize}

\section{Имплементация}
Симулацията е реализирана на езика Python с използване на библиотеката \texttt{simpy}.

\subsection{Основни параметри}
\begin{lstlisting}[language=Python, caption=Конфигурация на симулацията]
SIMULATION_HOURS = 4
SIMULATION_TIME = SIMULATION_HOURS * 3600  # 4 hours in seconds

# Sensor A: arrives every (9 \pm 4) seconds
SENSOR_A_MEAN = 9
SENSOR_A_VAR = 4

# Sensor B: arrives every 2 seconds
SENSOR_B_INTERVAL = 2

# Preprocessing
PREPROCESS_TIME = 10
PREPROCESS_FAILURE_RATE = 0.24  # 24\% failure

# Distribution
MESSAGES_PER_TASK = 4  # 4 messages per task for preprocessing
A_FOR_DISTRIBUTION = 2  # Need 2 type A messages
B_FOR_DISTRIBUTION = 3  # Need 3 type B messages

# Storage time
STORAGE_TIME = 0.5
\end{lstlisting}

\subsection{Генериране на съобщения}
\begin{lstlisting}[language=Python, caption=Генератори на сензори]
def sensor_a_generator(self):
    while True:
        interarrival = random.uniform(SENSOR_A_MEAN - SENSOR_A_VAR, 
                                     SENSOR_A_MEAN + SENSOR_A_VAR)
        yield self.env.timeout(interarrival)
        
        self.sensor_a_generated += 1
        # Add to task buffer for grouping
        self.task_buffer.append((msg_id, 'A', arrival_time))
        
        if len(self.task_buffer) >= 4:
            task = self.task_buffer[:4]
            self.task_buffer = self.task_buffer[4:]
            self.env.process(self.preprocess_task(task))
\end{lstlisting}

\subsection{Предварителна обработка с повторен опит}
\begin{lstlisting}[language=Python, caption=Обработка на задачи]
def preprocess_task(self, task_messages):
    # Randomly choose processor
    if random.random() < 0.5:
        processor = self.processor1
    else:
        processor = self.processor2
    
    with processor.request() as req:
        yield req
        yield self.env.timeout(10)
        
        # Check for 24% failure
        if random.random() < 0.24:
            # Retry
            yield self.env.timeout(10)
            self.tasks_retried += 1
        
        # On success, add to distribution buffers
        for msg_id, msg_type, _ in task_messages:
            if msg_type == 'A':
                self.preprocessed_a.append((msg_id, self.env.now))
            else:
                self.preprocessed_b.append((msg_id, self.env.now))
\end{lstlisting}

\subsection{Реализация на GPSS}
Алтернативна реализация на модела чрез езика за симулации GPSS.

\begin{lstlisting}[language=bash, caption=GPSS код на модела]
* Topic 9: Data Collection System - GPSS Model
GENERATE 9,4
ASSIGN 1,1
TRANSFER ,BUFFER

GENERATE 2
ASSIGN 1,2
TRANSFER ,BUFFER

GTASK SEIZE 10
ADVANCE 0
RELEASE 10
SPLIT 1,PROC1
TRANSFER ,PROC2

PROC1 SEIZE 1
ADVANCE 10
RELEASE 1
TRANSFER ,CHECK_FAIL

PROC2 SEIZE 2
ADVANCE 10
RELEASE 2
TRANSFER ,CHECK_FAIL

CHECK_FAIL SPLIT 1,RETRY
TRANSFER ,SUCCESS

RETRY SEIZE 3
ADVANCE 10
RELEASE 3

SUCCESS TERMINATE 0

* Timer - 4 hours = 14400 seconds
GENERATE 14400
TERMINATE 1
START 1
\end{lstlisting}

\subsection{Резултати от GPSS симулация}
{\footnotesize
\begin{lstlisting}[language=bash, basicstyle=\footnotesize\ttfamily]
              GPSS World Simulation Report - Untitled Model 1.1.1

                   Saturday, February 14, 2026 23:56:35  

           START TIME           END TIME  BLOCKS  FACILITIES  STORAGES
                0.000          14400.000    27        4          0


 LABEL              LOC  BLOCK TYPE     ENTRY COUNT CURRENT COUNT RETRY
                    1    GENERATE          1612             0       0
                    2    ASSIGN            1612             0       0
                    3    TRANSFER          1612             0       0
                    4    GENERATE          7200             0       0
                    5    ASSIGN            7200             0       0
                    6    TRANSFER          7200             0       0
GTASK               7    SEIZE             8812             0       0
                    8    ADVANCE           8812             0       0
                    9    RELEASE           8812             0       0
                   10    SPLIT             8812             0       0
                   11    TRANSFER          8812          7372       0
PROC1              12    SEIZE             1440             0       0
                   13    ADVANCE           1440             1       0
                   14    RELEASE           1439             0       0
                   15    TRANSFER          1439             0       0
PROC2              16    SEIZE             1440             0       0
                   17    ADVANCE           1440             1       0
                   18    RELEASE           1439             0       0
                   19    TRANSFER          1439             0       0
CHECK_FAIL         20    SPLIT             2878             0       0
                   21    TRANSFER          2878             0       0
RETRY              22    SEIZE             1439             0       0
                   23    ADVANCE           1439             1       0
                   24    RELEASE           1438             0       0
SUCCESS            25    TERMINATE         4316             0       0
                   26    GENERATE             1             0       0
                   27    TERMINATE            1             0       0


FACILITY         ENTRIES  UTIL.   AVE. TIME AVAIL. OWNER PEND INTER RETRY DELAY
 1                 1440    1.000       9.999  1     3351    0    0     0   7371
 2                 1440    1.000       9.999  1     3332    0    0     0   7372
 3                 1439    0.999       9.999  1    10260    0    0     0   1439
 10                8812    0.000       0.000  1        0    0    0     0      0
\end{lstlisting}
}

\section{Резултати от симулацията}
Проведена е симулация с продължителност 4 часа (14400 секунди).

\subsection{Статистически данни}
\begin{table}[H]
    \centering
    \caption{Резултати от симулацията}
    \begin{tabular}{|l|c|c|}
        \hline
        \textbf{Параметър} & \textbf{Python} & \textbf{GPSS} \\
        \hline
        Сензор A съобщения & 1591 & 1612 \\
        Сензор B съобщения & 7199 & 7200 \\
        Задачи обработени & 2177 & -- \\
        Неуспешни обработки & 533 & -- \\
        Дистрибуции завършени & 787 & -- \\
        Съобщения в хранилище & 3935 & -- \\
        \hline
    \end{tabular}
\end{table}

Както се вижда от таблицата, GPSS симулацията има проблеми с реализацията на групиране и дистрибуция, поради ограниченията на студентската версия.

\subsection{Сравнение между Python и GPSS}
За валидация на резултатите са реализирани две независими симулации - с езика Python (използвайки библиотеката \texttt{simpy}) и с езика GPSS (в GPSS World Student).

\begin{table}[H]
    \centering
    \caption{Сравнение на резултатите между Python и GPSS симулациите}
    \label{tab:comparison}
    \begin{tabular}{|l|c|c|}
        \hline
        \textbf{Тип съобщения} & \textbf{Python} & \textbf{GPSS} \\
        \hline
        Сензор A & 1591 & 1612 \\
        Сензор B & 7199 & 7200 \\
        Обработени единици & 2177 задачи & 8812 транзакции \\
        \hline
    \end{tabular}
\end{table}

Разликите се дължат на:
\begin{itemize}
    \item GPSS не реализира явно групиране на съобщения в задачи
    \item Различни реализации на равномерното разпределение
    \item Ограничения в GPSS Student при моделиране на сложни буфери
\end{itemize}

\subsection{Графична визуализация}
На фигурата по-долу е представена хистограма на разпределението на типовете съобщения.

\begin{figure}[H]
    \centering
    \begin{tikzpicture}
        \begin{axis}[
            width=12cm, height=8cm,
            xlabel={Тип съобщения},
            ylabel={Брой съобщения},
            title={Разпределение на съобщенията по тип},
            ybar,
            ymin=0,
            symbolic x coords={A,B},
            bar width=2cm
        ]
        % Simulation data
        \addplot coordinates {
            (A, 1591) (B, 7199)
        };
        \end{axis}
    \end{tikzpicture}
    \caption{Разпределение на съобщенията по сензори}
\end{figure}

\section{Заключение}
Симулационният модел показва, че системата успешно обработва входящите съобщения от двата типа сензори. Сензор B генерира значително повече съобщения (около 7200) в сравнение със сензор A (около 1600), което води до ограничение в дистрибуциите от страна на тип A съобщения.

Основните изводи:
\begin{enumerate}
    \item Около 2200 задачи са групирани и обработени
    \item 24\% от обработките се повтарят поради неуспех
    \item Тип A съобщения са ограничаващият фактор -- позволяват около 800 дистрибуции
    \item Около 3900 съобщения са успешно съхранени в двете хранилища
\end{enumerate}

За оптимизация на системата се препоръчва:
\begin{enumerate}
    \item Намаляване на честотата на сензор B или увеличаване на сензор A
    \item Добавяне на допълнителни процесори за предварителна обработка
    \item Оптимизиране на разпределителния модул за по-бързо съхранение
\end{enumerate}

Тези промени биха намалили времето за обработка с около 20-30\%.