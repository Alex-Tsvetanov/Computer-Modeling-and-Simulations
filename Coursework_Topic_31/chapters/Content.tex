\section{Постановка на задачата}
В компютър, работещ в система за управление на технологични процеси:
\begin{itemize}
    \item Информацията от сензорите се получава на всеки $(3 \pm 1)$ s.
    \item Продължителността на обработката е $(5 \pm 2)$ s.
    \item Съобщенията чакат в буферна памет.
    \item Динамиката на процеса изисква обработка само на съобщения, които изчакват не повече от 12 секунди в буферната памет. Останалите се считат за изгубени.
\end{itemize}
Необходимо е да се симулира процесът на получаване на 250 съобщения.

\section{Теоретична част}
Това е система за масово обслужване с ограничения (Deadline/Reneging).
Входящият поток е с равномерно разпределение в интервала $[2, 4]$ s (средно 3 s).
Времето за обслужване е равномерно разпределено в интервала $[3, 7]$ s (средно 5 s).

Тъй като средното време за обслужване (5 s) е по-голямо от средното време на постъпване (3 s), системата е претоварена ($\rho = 1.66$). Без ограничението от 12 секунди опашката би растяла безкрайно. Механизмът за отхвърляне на задачи (изгубени съобщения) стабилизира опашката, но води до загуба на информация.

\section{Имплементация}
Симулацията следи времето на пристигане на всяка заявка и проверява дали тя успява да получи достъп до процесора в рамките на 12 секунди.

\subsection{Код на симулацията}
\begin{lstlisting}[language=Python, caption=Логика за обработка и отхвърляне]
def process_message(env, name, server):
    arrival_time = env.now
    
    # Try to get server within DEADLINE
    with server.request() as req:
        results = yield req | env.timeout(DEADLINE)
        
        if req in results:
            # Server acquired within deadline
            wait_time = env.now - arrival_time
            monitor.wait_times.append(wait_time)
            
            # Processing
            proc_time = random.uniform(PROCESS_MIN, PROCESS_MAX)
            yield env.timeout(proc_time)
            monitor.processed_count += 1
        else:
            # Timeout reached (Wait > 12s)
            monitor.lost_count += 1
\end{lstlisting}

\subsection{Реализация на GPSS}
Алтернативна реализация на модела чрез езика за симулации GPSS.

\begin{lstlisting}[language=bash, caption=GPSS код на модела]
SIMULATE
GENERATE 3,1
MARK 1
SEIZE 1
TEST LE M1,12,Lost
ADVANCE 5,2
RELEASE 1
TERMINATE 1
Lost RELEASE 1
TERMINATE 1
START 250
\end{lstlisting}

\subsection{Резултати от GPSS симулация}
{\footnotesize
\begin{lstlisting}[language=bash, basicstyle=\footnotesize\ttfamily]
              GPSS World Simulation Report - Untitled Model 2.1.1


                   Saturday, February 14, 2026 22:32:17  

           START TIME           END TIME  BLOCKS  FACILITIES  STORAGES
                0.000            768.700     9        1          0

              NAME                       VALUE  
          LOST                            8.000

 LABEL              LOC  BLOCK TYPE     ENTRY COUNT CURRENT COUNT RETRY
                    1    GENERATE           255             0       0
                    2    MARK               255             4       0
                    3    SEIZE              251             1       0
                    4    TEST               250             0       0
                    5    ADVANCE            152             0       0
                    6    RELEASE            152             0       0
                    7    TERMINATE          152             0       0
LOST                8    RELEASE             98             0       0
                    9    TERMINATE           98             0       0

FACILITY         ENTRIES  UTIL.   AVE. TIME AVAIL. OWNER PEND INTER RETRY DELAY
 1                  251    0.995       3.048  1      251    0    0     0      4

CEC XN   PRI          M1      ASSEM  CURRENT  NEXT  PARAMETER    VALUE
   251    0         753.564    251      3      4       1        753.564

FEC XN   PRI         BDT      ASSEM  CURRENT  NEXT  PARAMETER    VALUE
   256    0         769.882    256      0      1
\end{lstlisting}
}

\section{Резултати от симулацията}
Симулацията приключи след генериране на 250 съобщения (прибл. 782 секунди).

\subsection{Обобщени данни}
\begin{table}[H]
    \centering
    \caption{Резултати за Тема 31 - сравнение Python и GPSS}
    \begin{tabular}{|l|r|r|}
        \hline
        \textbf{Показател} & \textbf{Python} & \textbf{GPSS} \\
        \hline
        Генерирани съобщения & 250 & 255 \\
        Успешно обработени & 155 & 152 \\
        Изгубени (Time-out) & 95 & 98 \\
        Процент загуби & 38.00\% & 39.22\% \\
        Средно време на чакане & 10.25 s & -- \\
        \hline
    \end{tabular}
\end{table}

Висок процент на загубите (38\%) се дължи на факта, че процесорът е значително по-бавен от входящия поток.

\subsection{Анализ на съответствието между Python и GPSS}
Резултатите от двете симулации (Python с \texttt{simpy} и GPSS World Student) показват много добро съответствие:

\begin{itemize}
    \item \textbf{Обработени съобщения}: Python (155) vs GPSS (152) -- разлика от само 2 съобщения (1.9\%)
    \item \textbf{Изгубени съобщения}: Python (95) vs GPSS (98) -- разлика от 3 съобщения (3.1\%)
    \item \textbf{Процент загуби}: Python (38.00\%) vs GPSS (39.22\%) -- разлика от 1.22 процентни пункта
\end{itemize}

Малките разлики се дължат на:
\begin{enumerate}
    \item Случайния характер на равномерните разпределения (различни seed-ове)
    \item Леко различната семантика на изчакването -- Python проверява крайния срок преди процеса, докато GPSS проверява след заемането на процесора
    \item GPSS симулацията генерира 255 съобщения вместо точно 250, което е особеност на терминирането в GPSS
\end{enumerate}

Въпреки тези разлики, и двете симулации потвърждават основния извод: системата губи приблизително 38-39\% от данните поради неспособността да обработи входния поток в рамките на зададения краен срок от 12 секунди.

\subsection{Визуализация на чакането}
Повечето обработени съобщения са чакали близо до максимално допустимото време (12 s), тъй като опашката е постоянно пълна.

\begin{figure}[H]
    \centering
    \begin{tikzpicture}
        \begin{axis}[
            width=10cm, height=6cm,
            xlabel={Номер на съобщение},
            ylabel={Време на чакане (s)},
            title={Време на чакане за успешно обработените съобщения},
            ymin=0, ymax=15,
            scatter, only marks
        ]
        % Simulation data (all points)
        \addplot coordinates {
            (1, 0.00) (2, 3.31) (3, 4.35) (4, 7.93) (5, 8.88) (6, 11.10) (7, 8.45) (8, 11.75) (9, 8.72) (10, 10.87) (11, 11.50) (12, 10.33) (13, 10.95) (14, 11.21) (15, 11.30) (16, 10.29) (17, 10.81) (18, 9.57) (19, 11.36) (20, 10.11) (21, 9.79) (22, 10.49) (23, 10.92) (24, 10.95) (25, 8.84) (26, 9.29) (27, 10.08) (28, 10.94) (29, 9.70) (30, 10.21) (31, 9.79) (32, 9.85) (33, 10.61) (34, 11.83) (35, 11.13) (36, 9.16) (37, 9.36) (38, 10.02) (39, 8.70) (40, 11.42) (41, 11.46) (42, 10.92) (43, 8.91) (44, 10.71) (45, 10.44) (46, 11.57) (47, 11.90) (48, 11.72) (49, 11.00) (50, 11.33) (51, 9.93) (52, 8.74) (53, 8.72) (54, 8.95) (55, 9.05) (56, 9.60) (57, 10.97) (58, 10.55) (59, 10.88) (60, 10.96) (61, 9.18) (62, 9.12) (63, 10.55) (64, 10.70) (65, 10.89) (66, 10.39) (67, 10.50) (68, 9.19) (69, 11.52) (70, 9.71) (71, 10.94) (72, 11.15) (73, 10.02) (74, 9.31) (75, 9.65) (76, 9.40) (77, 11.03) (78, 10.94) (79, 11.77) (80, 11.75) (81, 9.88) (82, 9.23) (83, 9.63) (84, 11.09) (85, 10.13) (86, 11.99) (87, 10.18) (88, 10.07) (89, 9.81) (90, 10.89) (91, 10.33) (92, 11.40) (93, 10.89) (94, 9.72) (95, 9.79) (96, 10.56) (97, 11.27) (98, 11.95) (99, 11.94) (100, 10.36) (101, 11.92) (102, 11.01) (103, 9.27) (104, 11.89) (105, 11.62) (106, 11.80) (107, 9.55) (108, 9.29) (109, 11.29) (110, 9.75) (111, 10.91) (112, 8.75) (113, 8.65) (114, 8.60) (115, 10.08) (116, 8.60) (117, 11.69) (118, 10.04) (119, 10.52) (120, 8.33) (121, 8.28) (122, 10.29) (123, 10.47) (124, 10.73) (125, 10.51) (126, 8.46) (127, 9.85) (128, 10.90) (129, 11.78) (130, 10.85) (131, 9.43) (132, 10.66) (133, 11.24) (134, 10.74) (135, 11.70) (136, 11.70) (137, 9.83) (138, 10.13) (139, 9.09) (140, 10.99) (141, 11.73) (142, 10.17) (143, 11.36) (144, 10.70) (145, 11.87) (146, 11.67) (147, 11.61) (148, 8.78) (149, 10.30) (150, 9.53) (151, 11.78) (152, 11.69) (153, 11.65) (154, 10.95) (155, 10.92)
        };
        \draw[red, dashed] (axis cs:0,12) -- (axis cs:160,12) node[anchor=south east] {Limit 12s};
        \end{axis}
    \end{tikzpicture}
    \caption{Динамика на времето за чакане}
\end{figure}

\section{Заключение}
Системата не е в състояние да обработи целия входящ поток данни поради ниската производителност на компютъра спрямо честотата на постъпване на данните. Загубата на 38\% от данните е значителна. За да се намалят загубите, е необходимо или да се повиши скоростта на обработка (напр. до 2-3 секунди средно), или да се въведе втори паралелен процесор.
