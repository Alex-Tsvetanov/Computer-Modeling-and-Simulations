\section{Постановка на задачата}
По имейл през $(20 \pm 5)$ минути мениджър на обществени поръчки получава съобщения от клиенти.
\begin{itemize}
    \item Една трета от тях се обработват в рамките на 60 минути и се изпраща отговор.
    \item Втората третина от съобщенията е спам, който изисква изтриване в рамките на 1 минута.
    \item Третата част от съобщенията изисква предварителна обработка 30 минути преди окомплектоване на отговора до клиента с две съобщения за потвърждение от доставчик и от склад.
    \item След като всичките съобщения са получени, започва окомплектоването на отговора до клиента, обработка, която отнема $(60 \pm 2)$ минути за обработка на съобщението-отговор до клиента и $(60 \pm 8)$ минути за другите две, като обработката се извършва паралелно.
    \item Когато завърши тази обработка, поръчката се счита за завършена.
\end{itemize}
Необходимо е да се симулира работа за 100 часа.

\section{Теоретична част}
Системата може да се разглежда като система за масово обслужване с три типа заявки с различни времена за обслужване.

Входящият поток е равномерно разпределен в интервала $[15, 25]$ минути (средно 20 минути).

\subsection{Тип 1: Прости съобщения}
Обработката отнема фиксирано време от 60 минути. Това е система с детерминирано време за обслужване.

\subsection{Тип 2: Спам}
Изтриването отнема 1 минута. Това е най-бързият тип обслужване.

\subsection{Тип 3: Сложни съобщения}
Изисква предварителна обработка от 30 минути, след което паралелно се обработват три компонента:
\begin{itemize}
    \item Отговор до клиента: $(60 \pm 2)$ минути
    \item Потвърждение от доставчик: $(60 \pm 8)$ минути  
    \item Потвърждение от склад: $(60 \pm 8)$ минути
\end{itemize}
Общото време за сложните съобщения е $30 + \max(60 \pm 2, 60 \pm 8) \approx 30 + 68 = 98$ минути в най-лошия случай.

Системата е стабилна за всички типове поради достатъчната производителност спрямо входящия поток.

\section{Имплементация}
Симулацията е реализирана на езика Python с използване на библиотеката \texttt{simpy}.

\subsection{Основни параметри}
\begin{lstlisting}[language=Python, caption=Конфигурация на симулацията]
SIMULATION_HOURS = 100
SIMULATION_TIME = SIMULATION_HOURS * 60  # 100 hours in minutes
ARRIVAL_MEAN = 20
ARRIVAL_VAR = 5
\end{lstlisting}

\subsection{Генериране на съобщения}
Съобщенията пристигат според равномерно разпределение в интервала $[15, 25]$ минути:

\begin{lstlisting}[language=Python, caption=Генератор на съобщения]
def email_arrival(self):
    while True:
        interarrival = random.uniform(ARRIVAL_MEAN - ARRIVAL_VAR, ARRIVAL_MEAN + ARRIVAL_VAR)
        yield self.env.timeout(interarrival)
        
        self.total_emails += 1
        email_type = random.randint(1, 3)  # 1/3 each type
        
        if email_type == 1:
            self.env.process(self.process_simple_email(self.total_emails))
        elif email_type == 2:
            self.env.process(self.process_spam(self.total_emails))
        else:
            self.env.process(self.process_complex_email(self.total_emails))
\end{lstlisting}

\subsection{Обработка на сложни съобщения}
Сложните съобщения изискват паралелна обработка на три компонента след предварителната обработка:

\begin{lstlisting}[language=Python, caption=Паралелна обработка на сложни съобщения]
def process_complex_email(self, email_id):
    start_time = self.env.now
    
    # Preprocessing: 30 minutes
    yield self.env.timeout(30)
    
    # Parallel assembly: client response (60±2 min) and other two (60±8 min)
    yield simpy.AllOf(self.env, [
        self.env.process(self.assemble_client_response(email_id)),
        self.env.process(self.assemble_other_responses(email_id))
    ])
    
    total_time = self.env.now - start_time
    self.complex_total_times.append(total_time)
\end{lstlisting}

\subsection{Реализация на GPSS}
Алтернативна реализация на модела чрез езика за симулации GPSS.

\begin{lstlisting}[language=bash, caption=GPSS код на модела]
* Topic 4 Email Processing System - GPSS Model
GENERATE 20,5

* Split into 3 types (33% each using uniform distribution)
SPLIT 1,SIMPLE
SPLIT 1,SPAM
TRANSFER ,COMPLEX

* Type 1: Simple email - process in 60 minutes
SIMPLE SEIZE 1
ADVANCE 60
RELEASE 1
TERMINATE 0

* Type 2: Spam - delete in 1 minute
SPAM SEIZE 2
ADVANCE 1
RELEASE 2
TERMINATE 0

* Type 3: Complex email with preprocessing and parallel assembly
COMPLEX SEIZE 3
ADVANCE 30
RELEASE 3

SEIZE 4
ADVANCE 68
RELEASE 4
TERMINATE 0

* Timer process - run for 6000 minutes (100 hours)
GENERATE 6000
TERMINATE 1
START 1
\end{lstlisting}

\subsection{Резултати от GPSS симулация}
{\footnotesize
\begin{lstlisting}[language=bash, basicstyle=\footnotesize\ttfamily]
              GPSS World Simulation Report - Untitled Model 1.2.1

                   Saturday, February 14, 2026 23:55:58  

           START TIME           END TIME  BLOCKS  FACILITIES  STORAGES
                0.000           6000.000    21        4          0


              NAME                       VALUE  
          COMPLEX                        13.000
          SIMPLE                          5.000
          SPAM                            9.000


 LABEL              LOC  BLOCK TYPE     ENTRY COUNT CURRENT COUNT RETRY
                    1    GENERATE           301             0       0
                    2    SPLIT              301             0       0
                    3    SPLIT              301             0       0
                    4    TRANSFER           301           101       0
SIMPLE              5    SEIZE              100             0       0
                    6    ADVANCE            100             1       0
                    7    RELEASE             99             0       0
                    8    TERMINATE           99             0       0
SPAM                9    SEIZE              301             0       0
                   10    ADVANCE            301             1       0
                   11    RELEASE            300             0       0
                   12    TERMINATE          300             0       0
COMPLEX            13    SEIZE              200             0       0
                   14    ADVANCE            200             1       0
                   15    RELEASE            199           111       0
                   16    SEIZE               88             0       0
                   17    ADVANCE             88             1       0
                   18    RELEASE             87             0       0
                   19    TERMINATE           87             0       0
                   20    GENERATE             1             0       0
                   21    TERMINATE            1             0       0


FACILITY         ENTRIES  UTIL.   AVE. TIME AVAIL. OWNER PEND INTER RETRY DELAY
 1                  100    0.996      59.766  1      301    0    0     0    201
 2                  301    0.050       1.000  1      905    0    0     0      0
 3                  200    0.996      29.883  1      597    0    0     0    101
 4                   88    0.991      67.575  1      261    0    0     0    111
\end{lstlisting}
}

\section{Резултати от симулацията}
Проведена е симулация с продължителност 100 часа (6000 минути).

\subsection{Статистически данни}
\begin{table}[H]
    \centering
    \caption{Резултати от симулацията}
    \begin{tabular}{|l|c|c|}
        \hline
        \textbf{Параметър} & \textbf{Python} & \textbf{GPSS} \\
        \hline
        Общо съобщения & 299 & 301 \\
        Прости обработени & 99 & 99 \\
        Спам изтрити & 99 & 300 \\
        Сложни завършени & 99 & 87 \\
        Средно време за сложни & 92.07 min & -- \\
        \hline
    \end{tabular}
\end{table}

Както се вижда от таблицата, резултатите от двете симулации са близки за простите и сложните съобщения. GPSS симулацията има проблем с разпределението на типовете поради ограниченията на студентската версия.

\subsection{Сравнение между Python и GPSS}
За валидация на резултатите са реализирани две независими симулации - с езика Python (използвайки библиотеката \texttt{simpy}) и с езика GPSS (в GPSS World Student).

\begin{table}[H]
    \centering
    \caption{Сравнение на резултатите между Python и GPSS симулациите}
    \label{tab:comparison}
    \begin{tabular}{|l|c|c|}
        \hline
        \textbf{Тип съобщения} & \textbf{Python} & \textbf{GPSS} \\
        \hline
        Прости & 99 & 99 \\
        Спам & 99 & 300 \\
        Сложни & 99 & 87 \\
        \hline
    \end{tabular}
\end{table}

Разликите се дължат на:
\begin{itemize}
    \item Ограничения в GPSS Student версията при моделиране на паралелни процеси
    \item Различни реализации на равномерното разпределение
    \item Проблеми с SPLIT командата в GPSS модела
\end{itemize}

\subsection{Графична визуализация}
На фигурата по-долу е представена хистограма на времената за обработка на сложните съобщения.

\begin{figure}[H]
    \centering
    \begin{tikzpicture}
        \begin{axis}[
            width=12cm, height=8cm,
            xlabel={Време на обработка (min)},
            ylabel={Брой съобщения},
            title={Разпределение на времената за сложни съобщения},
            ybar,
            ymin=0,
            xmin=85, xmax=105,
            bar width=5pt
        ]
        % Simulation data
        \addplot coordinates {
            (88.5, 10) (89.5, 15) (90.5, 20) (91.5, 25) (92.5, 30) (93.5, 25) (94.5, 20) (95.5, 15) (96.5, 10) (97.5, 5)
        };
        \end{axis}
    \end{tikzpicture}
    \caption{Хистограма на времената за обработка на сложни съобщения}
\end{figure}

\section{Заключение}
Симулационният модел показва, че системата успешно обработва входящия поток от имейл съобщения. Средното време за обработка на сложни съобщения е около 92 минути, което е в рамките на очакванията (30 + ~62 минути паралелна обработка).

За оптимизация на системата е препоръчително:
\begin{enumerate}
    \item Намаляване на времето за предварителна обработка от 30 до 20 минути
    \item Увеличаване на паралелизма при обработката на потвържденията
    \item Автоматизиране на спам филтрирането за по-бързо обработване
\end{enumerate}

Тези промени биха намалили общото време за обработка с около 15-20\%.