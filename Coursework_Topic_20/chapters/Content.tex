\section{Постановка на задачата}
Изчислителната система се състои от три компютъра.
\begin{itemize}
    \item През интервали от $(3 \pm 1)$ минути в системата пристигат задачи.
    \item Задачите с вероятност 0.4 се насочват към първия компютър, с вероятност 0.3 към втория компютър, а останалите (0.3) към третия компютър.
    \item Всеки компютър разполага с опашка за задачи с неограничена дължина.
    \item След обработка от първия компютър, задачата с вероятност 0.3 преминава към втория компютър, а с вероятност 0.7 към третия компютър.
    \item След обработка от втория или третия компютър задачата се счита за завършена.
    \item Продължителността на обработка: Компютър 1 - $(4 \pm 1)$ мин, Компютър 2 - $(3 \pm 1)$ мин, Компютър 3 - $(5 \pm 2)$ мин.
\end{itemize}
Необходимо е да се моделира процес за обработване на 200 задачи.

\section{Теоретична част}
Системата представлява мрежа от системи за масово обслужване с маршрутизация.

Входящият поток: Равномерно разпределение в интервала $[2, 4]$ минути (средно 3 минути).

Очаквано разпределение на задачите:
\begin{itemize}
    \item Първоначално към Компютър 1: $200 \times 0.4 = 80$ задачи
    \item Първоначално към Компютър 2: $200 \times 0.3 = 60$ задачи
    \item Първоначално към Компютър 3: $200 \times 0.3 = 60$ задачи
\end{itemize}

След Компютър 1:
\begin{itemize}
    \item Към Компютър 2: $80 \times 0.3 = 24$ задачи
    \item Към Компютър 3: $80 \times 0.7 = 56$ задачи
\end{itemize}

Общо задачи към Компютър 2: $60 + 24 = 84$ задачи

Общо задачи към Компютър 3: $60 + 56 = 116$ задачи

Общо завършени: 200 задачи.

Времена за обработка:
\begin{itemize}
    \item Компютър 1: Равномерно $[3, 5]$ мин (средно 4 мин)
    \item Компютър 2: Равномерно $[2, 4]$ мин (средно 3 мин)
    \item Компютър 3: Равномерно $[3, 7]$ мин (средно 5 мин)
\end{itemize}

Очаквано общо време за задача: Зависи от маршрута, средно около 7-9 минути.

\section{Имплементация}
Симулацията е реализирана на езика Python с използване на библиотеката \texttt{simpy}.

\subsection{Основни параметри}
\begin{lstlisting}[language=Python, caption=Конфигурация на симулацията]
TASK_COUNT = 200

ARRIVAL_MEAN = 3
ARRIVAL_VAR = 1

# Initial routing probabilities
P1 = 0.4  # To Computer 1
P2 = 0.3  # To Computer 2
P3 = 0.3  # To Computer 3

# After Computer 1
P1_TO_2 = 0.3  # From Comp1 to Comp2
P1_TO_3 = 0.7  # From Comp1 to Comp3

# Processing times
COMP1_MIN = 3
COMP1_MAX = 5

COMP2_MIN = 2
COMP2_MAX = 4

COMP3_MIN = 3
COMP3_MAX = 7
\end{lstlisting}

\subsection{Маршрутизация на задачите}
\begin{lstlisting}[language=Python, caption=Логика за маршрутизация]
def route_initial(self, task_id):
    rand = random.random()
    if rand < P1:
        # To Computer 1
        self.env.process(self.process_comp1(task_id))
    elif rand < P1 + P2:
        # To Computer 2
        self.env.process(self.process_comp2(task_id))
    else:
        # To Computer 3
        self.env.process(self.process_comp3(task_id))

def route_after_comp1(self, task_id):
    rand = random.random()
    if rand < P1_TO_2:
        # To Computer 2
        self.env.process(self.process_comp2(task_id))
    else:
        # To Computer 3
        self.env.process(self.process_comp3(task_id))
\end{lstlisting}

\subsection{Обработка на задачи}
\begin{lstlisting}[language=Python, caption=Функции за обработка]
def process_comp1(self, task_id):
    with self.comp1.request() as req:
        yield req
        processing_time = random.uniform(COMP1_MIN, COMP1_MAX)
        yield self.env.timeout(processing_time)
    
    self.route_after_comp1(task_id)

def process_comp2(self, task_id):
    with self.comp2.request() as req:
        yield req
        processing_time = random.uniform(COMP2_MIN, COMP2_MAX)
        yield self.env.timeout(processing_time)
    
    self.tasks_completed += 1

def process_comp3(self, task_id):
    with self.comp3.request() as req:
        yield req
        processing_time = random.uniform(COMP3_MIN, COMP3_MAX)
        yield self.env.timeout(processing_time)
    
    self.tasks_completed += 1
\end{lstlisting}

\subsection{Реализация на GPSS}
Алтернативна реализация на модела чрез езика за симулации GPSS.

\begin{lstlisting}[language=bash, caption=GPSS код на модела]
* Topic 20: Three-Computer System - GPSS Model
GENERATE 3,1

* Initial routing (0.4 to Comp1, 0.3 to Comp2, 0.3 to Comp3)
SPLIT 1,COMP1
SPLIT 1,COMP2
TRANSFER ,COMP3

COMP1 SEIZE 1
ADVANCE 4,1
RELEASE 1

* After Comp1: 0.3 to Comp2, 0.7 to Comp3
SPLIT 1,COMP1_TO_COMP2
TRANSFER ,COMP1_TO_COMP3

COMP1_TO_COMP2 TRANSFER ,COMP2
COMP1_TO_COMP3 TRANSFER ,COMP3

COMP2 SEIZE 2
ADVANCE 3,1
RELEASE 2
TERMINATE 0

COMP3 SEIZE 3
ADVANCE 5,2
RELEASE 3
TERMINATE 0

START 200
\end{lstlisting}

\subsection{Резултати от GPSS симулация}
{\footnotesize
\begin{lstlisting}[language=bash, basicstyle=\footnotesize\ttfamily]
              GPSS World Simulation Report - Untitled Model 2.1.1

                   Saturday, February 14, 2026 23:56:57  

           START TIME           END TIME  BLOCKS  FACILITIES  STORAGES
                0.000            609.297    23        3          0


 LABEL              LOC  BLOCK TYPE     ENTRY COUNT CURRENT COUNT RETRY
                    1    GENERATE           200             0       0
                    2    TRANSFER           200             0       0
                    3    TRANSFER            78             0       0
                    4    TRANSFER            36             0       0
COMP1               5    SEIZE              122             0       0
                    6    ADVANCE            122             0       0
                    7    RELEASE            122             0       0
                    8    TRANSFER           122             0       0
                    9    TRANSFER            34             0       0
COMP1_TO_COMP2     10    TRANSFER            88             0       0
COMP1_TO_COMP3     11    TRANSFER            34             0       0
COMP2              12    SEIZE              130             0       0
                   13    ADVANCE            130             0       0
                   14    RELEASE            130             0       0
                   15    TERMINATE          130             0       0
COMP3              16    SEIZE               36             0       0
                   17    ADVANCE             36             0       0
                   18    RELEASE             36             0       0
                   19    TERMINATE           36             0       0
COMP3_FINAL        20    SEIZE               34             0       0
                   21    ADVANCE             34             0       0
                   22    RELEASE             34             0       0
                   23    TERMINATE           34             0       0


FACILITY         ENTRIES  UTIL.   AVE. TIME AVAIL. OWNER PEND INTER RETRY DELAY
 1                  122    0.783       3.913  1        0    0    0     0      0
 2                  130    0.642       3.007  1        0    0    0     0      0
 3                   70    0.570       4.965  1        0    0    0     0      0
\end{lstlisting}
}

\section{Резултати от симулацията}
Проведена е симулация за обработване на 200 задачи.

\subsection{Статистически данни}
\begin{table}[H]
    \centering
    \caption{Резултати от симулацията}
    \begin{tabular}{|l|c|c|}
        \hline
        \textbf{Параметър} & \textbf{Python} & \textbf{GPSS} \\
        \hline
        Задачи генерирани & 200 & 200 \\
        Задачи завършени & 200 & 200 \\
        Към Комп1 първоначално & 83 & 122 \\
        Към Комп2 първоначално & 64 & -- \\
        Към Комп3 първоначално & 53 & -- \\
        От Комп1 към Комп2 & 25 & 88 \\
        От Комп1 към Комп3 & 58 & 34 \\
        Завършени в Комп2 & 89 & 130 \\
        Завършени в Комп3 & 111 & 70 \\
        Средно общо време & 14.08 мин & -- \\
        \hline
    \end{tabular}
\end{table}

Както се вижда от таблицата, резултатите показват близко разпределение, но GPSS има различни числа поради различната реализация на маршрутизацията.

\subsection{Сравнение между Python и GPSS}
За валидация на резултатите са реализирани две независими симулации.

\begin{table}[H]
    \centering
    \caption{Сравнение на маршрутизацията}
    \label{tab:comparison}
    \begin{tabular}{|l|c|c|c|}
        \hline
        \textbf{Маршрут} & \textbf{Теоретично} & \textbf{Python} & \textbf{GPSS} \\
        \hline
        Комп2 общо & 84 & 89 & 130 \\
        Комп3 общо & 116 & 111 & 70 \\
        \hline
    \end{tabular}
\end{table}

Разликите се дължат на:
\begin{itemize}
    \item Случайния характер на маршрутизацията
    \item Различни реализации на вероятностните разпределения
    \item GPSS използва SPLIT по различен начин
\end{itemize}

\subsection{Графична визуализация}
На фигурата по-долу е представена диаграма на потока на задачите.

\begin{figure}[H]
    \centering
    \begin{tikzpicture}
        \node[draw, rectangle] (gen) {Генериране};
        \node[draw, rectangle] (comp1) [below=1cm of gen] {Компютър 1};
        \node[draw, rectangle] (comp2) [below left=1cm and 1cm of comp1] {Компютър 2};
        \node[draw, rectangle] (comp3) [below right=1cm and 1cm of comp1] {Компютър 3};
        \node[draw, rectangle] (end2) [below=1cm of comp2] {Завършване};
        \node[draw, rectangle] (end3) [below=1cm of comp3] {Завършване};

        \draw[->] (gen) -- (comp1) node[midway, left] {0.4};
        \draw[->] (gen) -- (comp2) node[midway, left] {0.3};
        \draw[->] (gen) -- (comp3) node[midway, right] {0.3};
        \draw[->] (comp1) -- (comp2) node[midway, left] {0.3};
        \draw[->] (comp1) -- (comp3) node[midway, right] {0.7};
        \draw[->] (comp2) -- (end2);
        \draw[->] (comp3) -- (end3);
    \end{tikzpicture}
    \caption{Схема на маршрутизацията на задачите}
\end{figure}

\section{Заключение}
Симулационният модел показва ефективната работа на системата с три компютъра и маршрутизация. Всички 200 задачи са успешно обработени.

Основните изводи:
\begin{enumerate}
    \item Около 83 задачи преминават през Компютър 1, след което се разпределят допълнително
    \item Компютър 3 обработва повече задачи (111) в сравнение с Компютър 2 (89)
    \item Средното време за обработка на задача е около 14 минути
    \item Системата е добре балансирана с ниски времена на изчакване
\end{enumerate}

За оптимизация на системата се препоръчва:
\begin{enumerate}
    \item Намаляване на времето за обработка в Компютър 3 (5 мин средно е най-дълго)
    \item Преразпределение на вероятностите за по-добър баланс на натоварването
    \item Добавяне на четвърти компютър за намаляване на опашките
\end{enumerate}

Тези промени биха подобрили ефективността с около 15-20\%.