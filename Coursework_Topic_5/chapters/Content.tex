\section{Постановка на задачата}
Центърът за обработка на заявки се състои от два компютъра. 
\begin{itemize}
    \item Заявките към първия компютър се получават в съответствие с експоненциално разпределение, като средното време между пристигането на заявките е 1 секунда.
    \item Вторият компютър получава заявки в съответствие с нормално разпределение, като базовото време за обработка е 3 секунди, с максимално отклонение от 1 секунда.
    \item Обработката на заявка от всеки компютър продължава $(2 \pm 1)$ секунди.
\end{itemize}
Необходимо е да се моделира системата за период от половин час (1800 секунди).

\section{Теоретична част}
Системата може да се разглежда като съвкупност от две независими системи за масово обслужване (СМО).

\subsection{Първа система (Компютър 1)}
Входящият поток е Поасонов (интервалите между заявките са експоненциално разпределени) със средна стойност $\lambda^{-1} = 1$ s. Времето за обслужване е равномерно разпределено в интервала $[1, 3]$ s (тъй като $2 \pm 1$ s). Това е система от тип M/G/1.

Тъй като средното време за обслужване е $E[S] = 2$ s, а средното време между заявките е $E[A] = 1$ s, коефициентът на натоварване е $\rho = \frac{E[S]}{E[A]} = 2$. Тъй като $\rho > 1$, системата е нестабилна и дължината на опашката ще расте неограничено във времето.

\subsection{Втора система (Компютър 2)}
Входящият поток се характеризира с нормално разпределение на интервалите между заявките с $\mu_A = 3$ s и отклонение, което покрива диапазона $\pm 1$ s (приемаме $\sigma \approx 1/3$, за да може $3\sigma \approx 1$). Времето за обслужване е същото като при първия компютър: равномерно в $[1, 3]$ s.

Тук $E[S] = 2$ s, а $E[A] = 3$ s. Коефициентът на натоварване е $\rho = \frac{2}{3} \approx 0.67$. Тъй като $\rho < 1$, системата е стабилна.

\section{Имплементация}
Симулацията е реализирана на езика Python с използване на библиотеката \texttt{simpy}.

\subsection{Основни параметри}
\begin{lstlisting}[language=Python, caption=Конфигурация на симулацията]
SIMULATION_TIME = 1800  # 30 minutes

# Computer 1
ARRIVAL_MEAN_1 = 1.0
PROCESS_MIN_1 = 1.0
PROCESS_MAX_1 = 3.0

# Computer 2
ARRIVAL_MEAN_2 = 3.0
PROCESS_MIN_2 = 1.0
PROCESS_MAX_2 = 3.0
\end{lstlisting}

\subsection{Генериране на заявки}
Използват се два генератора за двата компютъра:

\begin{lstlisting}[language=Python, caption=Генератори на заявки]
def source_1(env, server):
    """Source for Computer 1: Exponential Inter-arrival"""
    while True:
        yield env.timeout(random.expovariate(1.0 / ARRIVAL_MEAN_1))
        env.process(process_request(env, server, ...))

def source_2(env, server):
    """Source for Computer 2: Normal Inter-arrival"""
    while True:
        dt = random.gauss(ARRIVAL_MEAN_2, 0.33) # Sigma approx 1/3
        if dt < 0: dt = 0
        yield env.timeout(dt)
        env.process(process_request(env, server, ...))
\end{lstlisting}

\subsection{Реализация на GPSS}
Алтернативна реализация на модела чрез езика за симулации GPSS.

\begin{lstlisting}[language=bash, caption=GPSS код на модела]
SIMULATE
GENERATE 1
SEIZE 1
ADVANCE 2,1
RELEASE 1
TERMINATE 0
GENERATE 3,1
SEIZE 2
ADVANCE 2,1
RELEASE 2
TERMINATE 0
GENERATE 1800
TERMINATE 1
START 1
\end{lstlisting}

\subsection{Резултати от GPSS симулация}
{\footnotesize
\begin{lstlisting}[language=bash, basicstyle=\footnotesize\ttfamily]
              GPSS World Simulation Report - Untitled Model 1.10.1


                   Saturday, February 14, 2026 22:30:26  

           START TIME           END TIME  BLOCKS  FACILITIES  STORAGES
                0.000           1800.000    12        2          0


 LABEL              LOC  BLOCK TYPE     ENTRY COUNT CURRENT COUNT RETRY
                    1    GENERATE          1800           890       0
                    2    SEIZE              910             0       0
                    3    ADVANCE            910             1       0
                    4    RELEASE            909             0       0
                    5    TERMINATE          909             0       0
                    6    GENERATE           595             0       0
                    7    SEIZE              595             0       0
                    8    ADVANCE            595             1       0
                    9    RELEASE            594             0       0
                   10    TERMINATE          594             0       0
                   11    GENERATE             1             0       0
                   12    TERMINATE            1             0       0


FACILITY         ENTRIES  UTIL.   AVE. TIME AVAIL. OWNER PEND INTER RETRY DELAY
 1                  910    0.999       1.977  1     1217    0    0     0    890
 2                  595    0.662       2.002  1     2391    0    0     0      0


FEC XN   PRI         BDT      ASSEM  CURRENT  NEXT  PARAMETER    VALUE
  2391    0        1800.170   2391      8      9
  1217    0        1800.523   1217      3      4
  2394    0        1800.785   2394      0      6
  2398    0        1801.000   2398      0      1
  2399    0        3600.000   2399      0     11
\end{lstlisting}
}

\section{Резултати от симулацията}
Проведена е симулация с продължителност 1800 секунди.

\subsection{Статистически данни}
\begin{table}[H]
    \centering
    \caption{Резултати от симулацията}
    \begin{tabular}{|l|c|c|}
        \hline
        \textbf{Параметър} & \textbf{Компютър 1} & \textbf{Компютър 2} \\
        \hline
        Обработени заявки & 895 & 597 \\
        Средно време на изчакване (s) & 444.78 & 0.016 \\
        Максимално време на изчакване (s) & 932.10 & 0.91 \\
        Натовареност & 99.93\% & 67.27\% \\
        \hline
    \end{tabular}
\end{table}

Както се очакваше теоретично, Компютър 1 е претоварен (почти 100\% утилизация и огромни времена на чакане), докато Компютър 2 работи в стабилен режим с минимално изчакване.

\subsection{Сравнение между Python и GPSS}
За валидация на резултатите са реализирани две независими симулации - с езика Python (използвайки библиотеката \texttt{simpy}) и с езика GPSS (в GPSS World Student). Резултатите от двете симулации са сравнени в Таблица \ref{tab:comparison}.

\begin{table}[H]
    \centering
    \caption{Сравнение на резултатите между Python и GPSS симулациите}
    \label{tab:comparison}
    \begin{tabular}{|l|c|c|c|c|}
        \hline
        \textbf{Параметър} & \multicolumn{2}{c|}{\textbf{Компютър 1}} & \multicolumn{2}{c|}{\textbf{Компютър 2}} \\
        \cline{2-5}
         & \textbf{Python} & \textbf{GPSS} & \textbf{Python} & \textbf{GPSS} \\
        \hline
        Обработени заявки & 895 & 910 & 597 & 595 \\
        Натовареност & 99.93\% & 99.9\% & 67.27\% & 66.2\% \\
        \hline
    \end{tabular}
\end{table}

Както се вижда от таблицата, резултатите от двете симулации са много близки. Малките разлики се дължат на:
\begin{itemize}
    \item Случайния характер на входните потоци (различни seed-ове)
    \item Леко различните реализации на експоненциалното разпределение
    \item Python симулацията използва пълна реализация на нормалното разпределение, докато GPSS симулацията използва равномерно разпределение за втория компютър (3$\pm$1s) поради ограниченията на студентската версия
\end{itemize}

Въпреки тези разлики, двете симулации потвърждават еднакво основните изводи:
\begin{enumerate}
    \item Компютър 1 е в нестабилен режим ($\rho > 1$) с неограничено растяща опашка
    \item Компютър 2 работи стабилно ($\rho < 1$) с минимални времена на чакане
\end{enumerate}

\subsection{Графична визуализация}
На фигурата по-долу е представена хистограма на времената за изчакване за Компютър 1. Вижда се, че времената за изчакване са разпределени в широк диапазон, което е характерно за нестабилна система.

\begin{figure}[H]
    \centering
    \begin{tikzpicture}
        \begin{axis}[
            width=12cm, height=8cm,
            xlabel={Време на изчакване (s)},
            ylabel={Брой заявки},
            title={Разпределение на времето за изчакване (Компютър 1)},
            ybar,
            ymin=0,
            xmin=0, xmax=1000,
            bar width=5pt
        ]
        % Simulation data
        \addplot coordinates {
            (46.6, 86) (139.8, 118) (233.0, 100) (326.2, 93) (419.4, 92) (512.7, 75) (605.9, 88) (699.1, 80) (792.3, 80) (885.5, 84)
        };
        \end{axis}
    \end{tikzpicture}
    \caption{Хистограма на времената за изчакване при Компютър 1}
\end{figure}

За Компютър 2 времената за изчакване са пренебрежимо малки (под 1 секунда).

\section{Заключение}
Симулационният модел потвърждава, че при зададените параметри първият компютър не успява да обслужи постъпващия поток от заявки, което води до натрупване на опашка. Вторият компютър се справя успешно със своето натоварване. За оптимизация на системата е препоръчително да се пренасочи част от товара на Компютър 1 към Компютър 2 или да се увеличи производителността на Компютър 1.
